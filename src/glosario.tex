% Acrónimos
%TFG
\newacronym[description={Graphic processing unit},longplural={unidades de procesamiento gráfico}]{gpu}{GPU}{unidad de procesamiento gráfico}
\newacronym{qos}{QoS}{Quality of Service}
\newacronym{dpi}{DPI}{Deep Packet Inspection}
\newacronym{dpdk}{DPDK}{Data Plane Development Kit}
\newacronym{dfa}{DFA}{Deterministic finite automaton}
\newacronym{hpet}{HPET}{High Precision Event Timer}
\newacronym{rtp}{RTP}{Real Time Protocol}
\newacronym[description={Field Programmable Gate Array},longplural={dispositivos de hardware programable}]{fpga}{FPGA}{dispositivo de hardware programable}
\newacronym[description={Network Interface Card},longplural={Interfaces de red}]{nic}{NIC}{Interfaz de red}
\newacronym[description={Núcleo lógico o Logical CORE}]{lcore}{LCORE}{logical core}

%TFM
\newacronym[description={Virtual Machine},longplural={máquinas virtuales}]{vm}{VM}{máquina virtual}
\newacronym[description={Virtual Network Function},longplural={funciones de red virtuales}]{nfv}{NFV}{función de red virtual}
\newacronym[description={Virtual Function},longplural={funciones virtuales}]{vf}{VF}{función virtual}
\newacronym[description={Internet Service Provider},longplural={proveedores de servicios de internet}]{isp}{ISP}{proveedor de servicios de Internet}
\newacronym[description={Software Defined Network},longplural={redes definidas por software}]{sdn}{SDN}{red definida por software}
\newacronym[description={Gigabit per second},longplural={Gigabits por segundo}]{gbps}{Gbps}{Gigabit por segundo}
\newacronym[description={Operative System},longplural={Sistemas operativos}]{os}{OS}{Sistema Operativo}
\newacronym[description={Solid-state drive},longplural={Discos de estado sólido}]{ssd}{SSD}{Disco de estado sólido}

% Glosario 

% TODO: Añadir aquí las definiciones del glosario
% Ejemplo de glosario
\newglossaryentry{cloud}{name={Cloud Computing},description={Se define la computación en la nube, como un conjunto de servicios que se encuentran en internet. Dichos servicios suelen estar descentralizados y repartidos en localizaciones y elementos físicos a los que usualmente no se tiene acceso}}

\newglossaryentry{cpu}{name={CPU},description={La tan conocida Unidad de procesamiento central, o CPU, se encarga de realizar la mayor parte de las tareas que realiza un ordenador de hoy en día. En las últimas arquitecturas, es también la encargada de coordinar diferentes dispositivos como la memoria o el bus PCIe.}}

\newglossaryentry{sriov}{name={SR-IOV},description={También conocida como \textit{Single Root I/O Virtualization}, SR-IOV~\cite{bib:introduccion:sriov} es una tecnología de virtualización que permite a un único dispositivo PCIe, mostrarse como diversas funciones virtuales (VF). Cada una de estas VF, puede conectarse a una máquina virtual, de forma que la compartición del dispositivo sea realizada por el propio hardware. Esto, minimiza el coste computacional de la máquina anfitriona, así como de las diversas máquinas virtuales que utilizan el dispositivo}}

\newglossaryentry{passthough}{name={Passthough},description={Se denomina Passthough, al mecanismo que tienen las máquinas virtuales para apoderarse por completo de un determinado dispositivo, ``desconectandolo'' de la máquina anfitriona. De esta forma, el sistema operativo invitado, es capaz de utilizar los drivers de forma nativa con el dispositivo y en principio con perdida de rendimiento 0}}

\newglossaryentry{virtio}{name={VirtIO},description={La tecnología VirtIO~\cite{russell2008virtio}, forma una parte importante de la tecnología de virtualización de KVM. El concepto de VirtIO, abarca desde los drivers de la máquina virtual, hasta los propios dispositivos virtuales generados por KVM. Un dispositivo VirtIO es, en esencia, una paravirtualización del dispositivo original}}

\newglossaryentry{mactiva}{name={monitorización activa},description={La monitorización activa consiste en la realización de una serie de medidas de una red, en donde dos o más nodos de la misma, intercambian un conjunto de paquetes. A partir de estos paquetes, se obtienen medidas como ancho de banda, jitter, latencia o cantidad de paquetes perdidos}}

\newglossaryentry{mpasiva}{name={monitorización pasiva},description={La monitorización pasiva consiste en la captura de tráfico en uno o varios puntos de la red. Analizando el tráfico capturado es posible obtener multitud de información, incluidos problemas a niveles de aplicación}}

\newglossaryentry{multicore}{name={multicore},description={Dentro de este contexto, la palabra \textit{multicore} hace referencia a los sitemas con más de una unidad de cómputo, ya sea porque hay más de un procesador completo, o se trate de un procesador con más de un \textit{core}}}

\newglossaryentry{core}{name={core},description={Entendemos como \textit{core}, como la unidad de procesamiento más básica de la que dispone nuestro sistema. Entendemos igualmente, que un \textit{core}, solo puede ejecutar un único hilo o proceso simultáneamente sin que su rendimiento se vea perjudicado}}


\newglossaryentry{zerocopy}{name={zero-copy},description={Hablamos de zero-copy, cuando una aplicación no necesita realizar ninguna copia de los datos (Normalmente de la memoria del kernel a la de usuario) para trabajar con ellos}}

\newglossaryentry{onecopy}{name={one-copy},description={Hablamos de one-copy, cuando una aplicación no requiere realizar una única copia de los datos (Normalmente de la memoria del kernel a la de usuario) para trabajar con ellos}}

\newglossaryentry{cache}{name={cache},description={Se suele llamar cache a una pequeña memoria de alta velocidad, que comunica una memoria lenta de tamaño muy superior y otro dispositivo. Por el principio de localidad y de temporalidad, almacenar los datos en una caché permite a, por ejemplo, microprocesadores, tener un acceso a los datos mucho más rápidamente}}

\newglossaryentry{vanilla}{name={vanilla},description={Se denomina a un elemento vanilla, cuando dicho elemento se encuentra en su estado original, sin modificaciones}}

\newglossaryentry{u}{name={U},description={Bajo el contexto de los equipos enracables, se denomina una U a la unidad básica de espacio que ocupa un equipo en un rack.}}

\newglossaryentry{raid0}{name={RAID 0},description={Se llama comunmente RAID a un grupo de discos duros que actúan en conjunto mostrándose al sistema operativo como un único disco duro. Un raid 0, es un caso específico en el cual todos los discos que forman el RAID, se encuentran al mismo nivel, es decir, no existe ningún disco que replique datos.}}

\newglossaryentry{hypervisor}{name={hypervisor},description={Se denomina hypervisor a la capa virtual que conecta una máquina virtual con los diferentes dispositivos físicos de la máquina anfitriona. Este elemento, también es el encargado de la paravirtualización y de la virtualización completa de dispositivos}}

%DOBLEGLOSARIOS:

\newglossaryentry{kvmg}{name={KVM},
    description={La tecnología KVM~\cite{bib:kvm} es una de las más populares en cuanto a virtualización se refiere, junto con su mayor competidor XEN. Esta tecnología se encuentra presente en el Kernel de Linux (en formato de módulo) y ofrece desde este nivel de ejecución la capacidad de virtualizar diferentes máquinas y dispositivos. Gracias a la tecnología actual de los procesadores (VT-X/AMD-V), esta virtualización puede llegar a ser transparente para la máquina virtualizada, además de ofrecer un aislamiento entre las diferentes máquinas virtuales y la propia máquina física}}
\newglossaryentry{kvm}{type=\acronymtype, name={KVM}, description={Kernel-based Virtual Machine}, first={máquina virtual basada en el Kernel (KVM)\glsadd{kvmg}}, see=[Glosario:]{kvmg}, firstplural={máquinas virtuales basadas en el Kernel (KVMs)\glsadd{kvmg}}}


\newglossaryentry{capexg}{name={CAPEX},
    description={El término inglés CAPEX, se refiere fundamentalmente a los gastos iniciales (o de actualización si es el caso), que son necesarios para poner en marcha un producto, o construir algo}}    
\newglossaryentry{capex}{type=\acronymtype, name={CAPEX}, description={CAPital EXpenditures}, first={gasto de capital inical (CAPEX)\glsadd{capexg}}, see=[Glosario:]{capexg}}


\newglossaryentry{opexg}{name={OPPEX},
    description={El término inglés OPPEX, se refiere fundamentalmente a los gastos que tiene un producto tras su contratación o compra}}    
\newglossaryentry{opex}{type=\acronymtype, name={OPPEX}, description={OPerational EXpenditures}, first={gasto de capital para operar (OPPEX)\glsadd{opexg}}, see=[Glosario:]{opexg}}


\newglossaryentry{numag}{name={NUMA},
    description={La arquitectura NUMA se basa en la descentralización de los recursos de un ordenador. De esta forma, cada procesador tiene su propia memoria y sus propios dispositivos. Si un procesador necesita acceder a los recursos de otro, debe pedirselos al procesador propietario de los mismos, el cual deberá interrumpir alguno de sus procesos para atender la petición. Estas peticiones, se realizan utilizando enlaces de altísima velocidad (como QPI), pero igualmente suelen ocasionar una degradación de rendimiento}}    
\newglossaryentry{numa}{type=\acronymtype, name={NUMA}, description={Non-Uniform Memory Access}, first={Acceso a Memoria No-Uniforme (NUMA)\glsadd{numag}}, see=[Glosario:]{numag}}


\newglossaryentry{mtug}{name={MTU},
    description={En el contexto de las redes de comunicaciones, se considera MTU al tamaño máximo que puede tener un paquete en un determinado enlace. Normalmente esta medida viene dada por una restricción del medio físico subyacente, no obstante, si no hubiese una limitación en la longitud de los paquetes, una aplicación podría abusar y emitir un paquete de tamaño infinito que colapsase indefinidamente la red}}    
\newglossaryentry{mtu}{type=\acronymtype, name={MTU}, description={Maximum Transmission Unit}, first={Unidad Máxima de Transmisión (MTU)\glsadd{mtug}}, see=[Glosario:]{mtug}}


\newglossaryentry{cpdg}{name={CPD},
    description={Un centro de datos (o Data Center), suele estar formado por varias habitaciones habilitadas para almacenar entre decenas y centenas de equipos. Estas salas se encuentran climatizadas, de forma que se mantenga una temperatura adecuada para el funcionamiento de los diferentes equipos. Estos centros de datos, suelen ofrecer a sus clientes luz y acceso a Internet a cambio de una cuota por cada U consumida}}    
\newglossaryentry{cpd}{type=\acronymtype, name={CPD}, description={Centro de procesamiento de datos}, first={Centro de procesamiento de datos (CPD)\glsadd{cpdg}}, see=[Glosario:]{cpdg}}


\newglossaryentry{hptlg}{name={HPTL},
    description={La librería HPTL partió de una primera implementación en mi trabajo fin de grado~\cite{rleira2013TFG}. Debido a la aparición de problemas de mantenibilidad y de configurabilidad, se decidió externalizar su uso como librería y publicarla en github~\cite{bib:hptl}}}    
\newglossaryentry{hptl}{type=\acronymtype, name={HPTL}, description={High Performance Timing Library}, first={librería de tiempo de alto rendimiento (HPTL)\glsadd{hptlg}}, see=[Glosario:]{hptlg}}


